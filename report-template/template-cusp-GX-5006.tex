\documentclass[10pt,twocolumn]{article} 
\usepackage{times}
\usepackage{graphicx}
\usepackage{amssymb}
\usepackage{titling}
\usepackage{url,hyperref}

\begin{document}

\title{\LaTeX\ Template for Assigment Reports}

\author{Beyonce and Laidy Gaga}

\date{%
CUSP-GX-5006\\
Assigment \# 1\\
\rule{\textwidth}{1pt}
}

\posttitle{\par\rule{3in}{0.4pt}\end{center}\vskip 0.5em}
%\postdate{\rule{\textwidth}{1pt}}

\maketitle

\begin{abstract}
Include a summary of the developments and their achievements here.
\end{abstract}


\section{Introduction}

Describe the main goals of your analysis and what you did 
achieve them. 

You can highlight some interesting findings and 
strategies you have adopted to tackle to assignment. 

\section{Methods and Data Sets}
Provide here a detailed description of the pipeline/steps involved in your experiment,
including:

\begin{enumerate}
\item A description of the data sets used in your experiments. You should detail:
\begin{itemize}
\item The number of variables involved, providing a short description of their meaning.
\item If some filtering was used to remove outliers or other artifacts from the data.
In affirmative case, clearly describe how the date was filtered.
\item Whether data was normalized or not.
\end{itemize}
\item A discussion about the computational methods used in your experiments and your 
their parameters where set
\item If your experiments relied on some particular procedure described in a particular 
book of scientific paper, you should cite the reference like this \cite{ref1,ref2}.
\end{enumerate}

\section{Results}

Describe your results and finds here. You can include figures to illustrate your results
and depict the findings. Figures can be pointed using the labels ~\ref{fig-cusp}. 

\begin{figure}[!t]
  \begin{center}
    \includegraphics[width=2.5in]{cusp.png}
  \end{center}

  \caption{\small Figures can be included like the one above}
  \label{fig-cusp}
\end{figure}

\begin{figure*}[!t]
  \begin{center}
    \includegraphics[width=6.5in]{cusp.png}
  \end{center}

  \caption{\small If you have a large figure you can include it as two columens}
  \label{fig-twocolumn}
\end{figure*}

\noindent\textbf{PS 1.} The file \texttt{refs.bib} contains the bibliographic references (see example).\\

\noindent\textbf{PS 2.} To generate a pdf you should run
\begin{enumerate}
\item pdflatex <file-name.tex>
\item bibtex <file-name> (without .tex)
\item pdflatex <file-name.tex>
\item pdflatex <file-name.tex>
\end{enumerate}


\bibliographystyle{abbrv}
\bibliography{refs}
\end{document}
